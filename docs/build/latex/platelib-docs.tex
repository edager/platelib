%% Generated by Sphinx.
\def\sphinxdocclass{report}
\documentclass[letterpaper,10pt,english]{sphinxmanual}
\ifdefined\pdfpxdimen
   \let\sphinxpxdimen\pdfpxdimen\else\newdimen\sphinxpxdimen
\fi \sphinxpxdimen=.75bp\relax

\PassOptionsToPackage{warn}{textcomp}
\usepackage[utf8]{inputenc}
\ifdefined\DeclareUnicodeCharacter
 \ifdefined\DeclareUnicodeCharacterAsOptional
  \DeclareUnicodeCharacter{"00A0}{\nobreakspace}
  \DeclareUnicodeCharacter{"2500}{\sphinxunichar{2500}}
  \DeclareUnicodeCharacter{"2502}{\sphinxunichar{2502}}
  \DeclareUnicodeCharacter{"2514}{\sphinxunichar{2514}}
  \DeclareUnicodeCharacter{"251C}{\sphinxunichar{251C}}
  \DeclareUnicodeCharacter{"2572}{\textbackslash}
 \else
  \DeclareUnicodeCharacter{00A0}{\nobreakspace}
  \DeclareUnicodeCharacter{2500}{\sphinxunichar{2500}}
  \DeclareUnicodeCharacter{2502}{\sphinxunichar{2502}}
  \DeclareUnicodeCharacter{2514}{\sphinxunichar{2514}}
  \DeclareUnicodeCharacter{251C}{\sphinxunichar{251C}}
  \DeclareUnicodeCharacter{2572}{\textbackslash}
 \fi
\fi
\usepackage{cmap}
\usepackage[T1]{fontenc}
\usepackage{amsmath,amssymb,amstext}
\usepackage{babel}
\usepackage{times}
\usepackage[Bjarne]{fncychap}
\usepackage{sphinx}

\usepackage{geometry}

% Include hyperref last.
\usepackage{hyperref}
% Fix anchor placement for figures with captions.
\usepackage{hypcap}% it must be loaded after hyperref.
% Set up styles of URL: it should be placed after hyperref.
\urlstyle{same}

\addto\captionsenglish{\renewcommand{\figurename}{Fig.}}
\addto\captionsenglish{\renewcommand{\tablename}{Table}}
\addto\captionsenglish{\renewcommand{\literalblockname}{Listing}}

\addto\captionsenglish{\renewcommand{\literalblockcontinuedname}{continued from previous page}}
\addto\captionsenglish{\renewcommand{\literalblockcontinuesname}{continues on next page}}

\addto\extrasenglish{\def\pageautorefname{page}}

\setcounter{tocdepth}{1}



\title{platelib Documentation}
\date{May 03, 2018}
\release{0.1.4-alpha}
\author{Emil Dandanell Agerschou}
\newcommand{\sphinxlogo}{\vbox{}}
\renewcommand{\releasename}{Release}
\makeindex

\begin{document}

\maketitle
\sphinxtableofcontents
\phantomsection\label{\detokenize{index::doc}}



\chapter{Introduction}
\label{\detokenize{introduction:introduction}}\label{\detokenize{introduction::doc}}\label{\detokenize{introduction:welcome-to-the-platelib-documentation}}

\section{Summary}
\label{\detokenize{introduction:summary}}
platelib is an attempt to make common tasks when working
with kinetic platereader, especially amyloid aggregation,
data easy and compatible with Python.


\section{Disclaimer}
\label{\detokenize{introduction:disclaimer}}
The state of this repository is one of very early development,
the code is not elegant, there most likely are bugs so
\sphinxstylestrong{Use with caution!}


\section{Read the docs}
\label{\detokenize{introduction:read-the-docs}}
A brief overview is given below, for more detailed information
see the docs directory and subfolders,
or even better have a look at the source :)


\chapter{Installation}
\label{\detokenize{install:installation}}\label{\detokenize{install::doc}}

\section{Prerequisites}
\label{\detokenize{install:prerequisites}}\begin{itemize}
\item {} 
\sphinxhref{https://www.python.org/downloads/}{python}

\item {} 
\sphinxhref{https://pip.pypa.io/en/stable/installing/}{pip}

\item {} 
\sphinxhref{https://git-scm.com/downloads}{git} (optional)

\end{itemize}


\section{Download}
\label{\detokenize{install:download}}\label{\detokenize{install:git}}
Either download the \sphinxhref{https://github.com/edager/platelib/tree/master/dist}{source distribution} directly or use:
\begin{quote}

\sphinxcode{\sphinxupquote{git https://github.com/edager/platelib}}
\end{quote}


\section{Install}
\label{\detokenize{install:source-distribution}}\label{\detokenize{install:install}}
Go to directory where the platelib source file is (\sphinxcode{\sphinxupquote{platelib/dist/}} if \sphinxcode{\sphinxupquote{git}} was used) and run the following in the terminal:

\fvset{hllines={, ,}}%
\begin{sphinxVerbatim}[commandchars=\\\{\}]
\PYG{n}{pip} \PYG{n}{install} \PYG{n}{platelib}\PYG{o}{\PYGZhy{}}\PYG{n}{X}\PYG{o}{.}\PYG{n}{X}\PYG{o}{.}\PYG{n}{tar}\PYG{o}{.}\PYG{n}{gz}
\end{sphinxVerbatim}

Where \sphinxcode{\sphinxupquote{X.X}} should be replaced by the version that was downloaded.


\section{Upgrade}
\label{\detokenize{install:upgrade}}
Go to directory where the platelib source file is (\sphinxcode{\sphinxupquote{platelib/dist/}} if \sphinxcode{\sphinxupquote{git}} was used) and run the following in the terminal:

\fvset{hllines={, ,}}%
\begin{sphinxVerbatim}[commandchars=\\\{\}]
\PYG{n}{pip} \PYG{n}{install} \PYG{o}{\PYGZhy{}}\PYG{o}{\PYGZhy{}}\PYG{n}{upgrade} \PYG{n}{platelib}\PYG{o}{\PYGZhy{}}\PYG{n}{X}\PYG{o}{.}\PYG{n}{X}\PYG{o}{.}\PYG{n}{tar}\PYG{o}{.}\PYG{n}{gz}
\end{sphinxVerbatim}

Where \sphinxcode{\sphinxupquote{X.X}} should be replaced by the version that was downloaded.


\section{Uninstall}
\label{\detokenize{install:uninstall}}
Simply go to the terminal and run:

\fvset{hllines={, ,}}%
\begin{sphinxVerbatim}[commandchars=\\\{\}]
\PYG{n}{pip} \PYG{n}{uninstall} \PYG{n}{platelib}
\end{sphinxVerbatim}


\chapter{Support}
\label{\detokenize{support:support}}\label{\detokenize{support::doc}}\begin{itemize}
\item {} \begin{description}
\item[{For questions related to usage:}] \leavevmode\begin{itemize}
\item {} 
\sphinxhref{https://stackoverflow.com/}{Stack Overflow} using the tags \sphinxcode{\sphinxupquote{python}} and \sphinxcode{\sphinxupquote{platelib}}

\end{itemize}

\end{description}

\item {} \begin{description}
\item[{For questions related to bugs or enhancements:}] \leavevmode\begin{itemize}
\item {} 
\sphinxhref{https://github.com/}{GitHub} using the issue system

\end{itemize}

\end{description}

\end{itemize}


\chapter{Cookbook}
\label{\detokenize{cookbook:github}}\label{\detokenize{cookbook::doc}}\label{\detokenize{cookbook:cookbook}}

\section{Reading in data}
\label{\detokenize{cookbook:reading-in-data}}
The main functionality of \sphinxcode{\sphinxupquote{platelib}} is the \sphinxcode{\sphinxupquote{read\_plate}} function
that allows for reading in platereader data from kinetic experiments
into a common framework namely into the \sphinxcode{\sphinxupquote{Plate\_data}} class.

If an equal number of replicates per sample were prepared
this can be specified (default is \sphinxcode{\sphinxupquote{3}}):

\fvset{hllines={, ,}}%
\begin{sphinxVerbatim}[commandchars=\\\{\}]
\PYG{n}{p} \PYG{o}{=} \PYG{n}{read\PYGZus{}plate}\PYG{p}{(}\PYG{l+s+s1}{\PYGZsq{}}\PYG{l+s+s1}{path/to/file}\PYG{l+s+s1}{\PYGZsq{}}\PYG{p}{,} \PYG{n}{replicates}\PYG{o}{=}\PYG{l+m+mi}{5}\PYG{p}{)}
\end{sphinxVerbatim}

It can be specified which direction the replicates were loaded
onto the plate where \sphinxcode{\sphinxupquote{'hori'}} (horizontal) means towards
increasing numbers and \sphinxcode{\sphinxupquote{'vert'}} is towards increasing letters
(default is \sphinxcode{\sphinxupquote{'hori'}}):

\fvset{hllines={, ,}}%
\begin{sphinxVerbatim}[commandchars=\\\{\}]
\PYG{n}{p} \PYG{o}{=} \PYG{n}{read\PYGZus{}plate}\PYG{p}{(}\PYG{l+s+s1}{\PYGZsq{}}\PYG{l+s+s1}{path/to/file}\PYG{l+s+s1}{\PYGZsq{}}\PYG{p}{,} \PYG{n}{rep\PYGZus{}direction}\PYG{o}{=}\PYG{l+s+s1}{\PYGZsq{}}\PYG{l+s+s1}{vert}\PYG{l+s+s1}{\PYGZsq{}}\PYG{p}{)}
\end{sphinxVerbatim}

\sphinxstylestrong{NOTE that the replicates have to be next to each other!}

Alternatively it can be specified which wells contains replicates:

\fvset{hllines={, ,}}%
\begin{sphinxVerbatim}[commandchars=\\\{\}]
\PYG{n}{p} \PYG{o}{=} \PYG{n}{read\PYGZus{}plate}\PYG{p}{(}\PYG{l+s+s1}{\PYGZsq{}}\PYG{l+s+s1}{path/to/file}\PYG{l+s+s1}{\PYGZsq{}}\PYG{p}{,} \PYG{n}{named\PYGZus{}samples}\PYG{o}{=}\PYG{p}{[}\PYG{p}{[}\PYG{l+s+s1}{\PYGZsq{}}\PYG{l+s+s1}{B03}\PYG{l+s+s1}{\PYGZsq{}}\PYG{p}{,} \PYG{l+s+s1}{\PYGZsq{}}\PYG{l+s+s1}{D07}\PYG{l+s+s1}{\PYGZsq{}}\PYG{p}{]}\PYG{p}{,} \PYG{p}{[}\PYG{l+s+s1}{\PYGZsq{}}\PYG{l+s+s1}{B02}\PYG{l+s+s1}{\PYGZsq{}}\PYG{p}{,} \PYG{l+s+s1}{\PYGZsq{}}\PYG{l+s+s1}{E06}\PYG{l+s+s1}{\PYGZsq{}}\PYG{p}{,} \PYG{l+s+s1}{\PYGZsq{}}\PYG{l+s+s1}{G12}\PYG{l+s+s1}{\PYGZsq{}}\PYG{p}{]}\PYG{p}{]}
\end{sphinxVerbatim}

Data from Tecan platereaders can be read in as (default is \sphinxcode{\sphinxupquote{'bmg'}}):

\fvset{hllines={, ,}}%
\begin{sphinxVerbatim}[commandchars=\\\{\}]
\PYG{n}{p} \PYG{o}{=} \PYG{n}{read\PYGZus{}plate}\PYG{p}{(}\PYG{l+s+s1}{\PYGZsq{}}\PYG{l+s+s1}{path/to/file}\PYG{l+s+s1}{\PYGZsq{}}\PYG{p}{,} \PYG{n}{platereader}\PYG{o}{=}\PYG{l+s+s1}{\PYGZsq{}}\PYG{l+s+s1}{tecan}\PYG{l+s+s1}{\PYGZsq{}}\PYG{p}{)}
\end{sphinxVerbatim}

\sphinxstylestrong{NOTE that this functionality has not been fully tested yet!}

As well as from BMG platereaders either where the data has prior
been transposed \sphinxcode{\sphinxupquote{True}} such that well data are in column format
or in row format \sphinxcode{\sphinxupquote{False}} (default is \sphinxcode{\sphinxupquote{True}}):

\fvset{hllines={, ,}}%
\begin{sphinxVerbatim}[commandchars=\\\{\}]
\PYG{n}{p} \PYG{o}{=} \PYG{n}{read\PYGZus{}plate}\PYG{p}{(}\PYG{l+s+s1}{\PYGZsq{}}\PYG{l+s+s1}{path/to/file}\PYG{l+s+s1}{\PYGZsq{}}\PYG{p}{,} \PYG{n}{transposed}\PYG{o}{=}\PYG{k+kc}{False}\PYG{p}{)}
\end{sphinxVerbatim}

Note that it’s automatically detected if several measurements
(\sphinxstyleemphasis{e.g.}) were made per time-point (see {\hyperref[\detokenize{cookbook:accessing-data}]{\sphinxcrossref{\DUrole{std,std-ref}{Accessing data}}}})

The time unit can also be specified which as either \sphinxcode{\sphinxupquote{'seconds'}},
\sphinxcode{\sphinxupquote{'minutes'}}, \sphinxcode{\sphinxupquote{'hours'}}, or \sphinxcode{\sphinxupquote{'days'}} will carry along into indexes
if exported and to unit of x-axis if plotted (default is \sphinxcode{\sphinxupquote{'hours'}}):

\fvset{hllines={, ,}}%
\begin{sphinxVerbatim}[commandchars=\\\{\}]
\PYG{n}{p} \PYG{o}{=} \PYG{n}{read\PYGZus{}plate}\PYG{p}{(}\PYG{l+s+s1}{\PYGZsq{}}\PYG{l+s+s1}{path/to/file}\PYG{l+s+s1}{\PYGZsq{}}\PYG{p}{,}\PYG{n}{time\PYGZus{}unit}\PYG{o}{=}\PYG{l+s+s1}{\PYGZsq{}}\PYG{l+s+s1}{days}\PYG{l+s+s1}{\PYGZsq{}}\PYG{p}{)}
\end{sphinxVerbatim}


\section{Accessing data}
\label{\detokenize{cookbook:accessing-data}}
The \sphinxcode{\sphinxupquote{Plate\_data}} class allows for different ways of accessing the data

Through index:

\fvset{hllines={, ,}}%
\begin{sphinxVerbatim}[commandchars=\\\{\}]
\PYG{n}{p}\PYG{p}{[}\PYG{l+m+mi}{1}\PYG{p}{]}
\end{sphinxVerbatim}

Through index slice:

\fvset{hllines={, ,}}%
\begin{sphinxVerbatim}[commandchars=\\\{\}]
\PYG{n}{p}\PYG{p}{[}\PYG{p}{:}\PYG{p}{:}\PYG{l+m+mi}{3}\PYG{p}{]}
\end{sphinxVerbatim}

Through well name:

\fvset{hllines={, ,}}%
\begin{sphinxVerbatim}[commandchars=\\\{\}]
\PYG{n}{p}\PYG{p}{[}\PYG{l+s+s1}{\PYGZsq{}}\PYG{l+s+s1}{B02}\PYG{l+s+s1}{\PYGZsq{}}\PYG{p}{]}
\end{sphinxVerbatim}

Through list of well names:

\fvset{hllines={, ,}}%
\begin{sphinxVerbatim}[commandchars=\\\{\}]
\PYG{n}{p}\PYG{p}{[}\PYG{p}{[}\PYG{l+s+s1}{\PYGZsq{}}\PYG{l+s+s1}{B02}\PYG{l+s+s1}{\PYGZsq{}}\PYG{p}{,}\PYG{l+s+s1}{\PYGZsq{}}\PYG{l+s+s1}{C03}\PYG{l+s+s1}{\PYGZsq{}}\PYG{p}{,}\PYG{l+s+s1}{\PYGZsq{}}\PYG{l+s+s1}{D04}\PYG{l+s+s1}{\PYGZsq{}}\PYG{p}{]}\PYG{p}{]}
\end{sphinxVerbatim}

Retrieved as a pandas.DataFrame with wellnames as column names and
time points as index:

\fvset{hllines={, ,}}%
\begin{sphinxVerbatim}[commandchars=\\\{\}]
\PYG{n}{df} \PYG{o}{=} \PYG{n}{Plate\PYGZus{}data}\PYG{o}{.}\PYG{n}{to\PYGZus{}a\PYGZus{}dataframe}\PYG{p}{(}\PYG{p}{)}
\end{sphinxVerbatim}

Or as a (C)omma (S)eperated (V)aribles file with the first line
being (time unit + ) well names and the first column are the
time points:

\fvset{hllines={, ,}}%
\begin{sphinxVerbatim}[commandchars=\\\{\}]
\PYG{n}{Plate\PYGZus{}data}\PYG{o}{.}\PYG{n}{to\PYGZus{}a\PYGZus{}csv}\PYG{p}{(}\PYG{l+s+s1}{\PYGZsq{}}\PYG{l+s+s1}{path/to/file.csv}\PYG{l+s+s1}{\PYGZsq{}}\PYG{p}{)}
\end{sphinxVerbatim}


\section{Plotting data}
\label{\detokenize{cookbook:plotting-data}}
The data is plotted according to replicates, and subtitles can be added
(default is \sphinxtitleref{None}):

\fvset{hllines={, ,}}%
\begin{sphinxVerbatim}[commandchars=\\\{\}]
\PYG{n}{p}\PYG{o}{.}\PYG{n}{plot}\PYG{p}{(}\PYG{n}{titles}\PYG{o}{=}\PYG{p}{[}\PYG{l+s+s1}{\PYGZsq{}}\PYG{l+s+s1}{condition 1}\PYG{l+s+s1}{\PYGZsq{}}\PYG{p}{,} \PYG{l+s+s1}{\PYGZsq{}}\PYG{l+s+s1}{conditions 2}\PYG{l+s+s1}{\PYGZsq{}}\PYG{p}{]}\PYG{p}{)}
\end{sphinxVerbatim}

It can be specified whether all plots should have its own y-axis,
whether all plots should have the same (default is \sphinxtitleref{True}):

\fvset{hllines={, ,}}%
\begin{sphinxVerbatim}[commandchars=\\\{\}]
\PYG{n}{p}\PYG{o}{.}\PYG{n}{plot}\PYG{p}{(}\PYG{n}{sharey}\PYG{o}{=}\PYG{l+s+s1}{\PYGZsq{}}\PYG{l+s+s1}{False}\PYG{l+s+s1}{\PYGZsq{}}\PYG{p}{)}
\end{sphinxVerbatim}

If several measurements were made per time-point it can be
specified whether all measurements should be plotted or not
(default is \sphinxtitleref{True}):

\fvset{hllines={, ,}}%
\begin{sphinxVerbatim}[commandchars=\\\{\}]
\PYG{n}{p}\PYG{o}{.}\PYG{n}{plot}\PYG{p}{(}\PYG{n}{plot\PYGZus{}multi}\PYG{o}{=}\PYG{l+s+s1}{\PYGZsq{}}\PYG{l+s+s1}{False}\PYG{l+s+s1}{\PYGZsq{}}\PYG{p}{)}
\end{sphinxVerbatim}


\chapter{platelib}
\label{\detokenize{modules:platelib}}\label{\detokenize{modules::doc}}

\section{platelib package}
\label{\detokenize{platelib::doc}}\label{\detokenize{platelib:platelib-package}}

\subsection{Submodules}
\label{\detokenize{platelib:submodules}}

\subsection{platelib.fitfun module}
\label{\detokenize{platelib:module-platelib.fitfun}}\label{\detokenize{platelib:platelib-fitfun-module}}\index{platelib.fitfun (module)}\index{exp\_rise() (in module platelib.fitfun)}

\begin{fulllineitems}
\phantomsection\label{\detokenize{platelib:platelib.fitfun.exp_rise}}\pysiglinewithargsret{\sphinxcode{\sphinxupquote{platelib.fitfun.}}\sphinxbfcode{\sphinxupquote{exp\_rise}}}{\emph{t}, \emph{a}, \emph{b}, \emph{k}}{}
\end{fulllineitems}

\index{fit\_fun() (in module platelib.fitfun)}

\begin{fulllineitems}
\phantomsection\label{\detokenize{platelib:platelib.fitfun.fit_fun}}\pysiglinewithargsret{\sphinxcode{\sphinxupquote{platelib.fitfun.}}\sphinxbfcode{\sphinxupquote{fit\_fun}}}{\emph{func, df, bounds=({[}0, 0, 0, 0{]}, {[}1, 50, 10, 65{]})}}{}
\end{fulllineitems}

\index{fit\_plate (class in platelib.fitfun)}

\begin{fulllineitems}
\phantomsection\label{\detokenize{platelib:platelib.fitfun.fit_plate}}\pysiglinewithargsret{\sphinxbfcode{\sphinxupquote{class }}\sphinxcode{\sphinxupquote{platelib.fitfun.}}\sphinxbfcode{\sphinxupquote{fit\_plate}}}{\emph{data}, \emph{replicates=3}, \emph{rep\_direction='hori'}, \emph{multi\_chrom=1}}{}
Bases: {\hyperref[\detokenize{platelib:platelib.plateread.Plate_data}]{\sphinxcrossref{\sphinxcode{\sphinxupquote{platelib.plateread.Plate\_data}}}}}

\end{fulllineitems}

\index{gauss() (in module platelib.fitfun)}

\begin{fulllineitems}
\phantomsection\label{\detokenize{platelib:platelib.fitfun.gauss}}\pysiglinewithargsret{\sphinxcode{\sphinxupquote{platelib.fitfun.}}\sphinxbfcode{\sphinxupquote{gauss}}}{\emph{x}, \emph{amp}, \emph{cen}, \emph{sigma}}{}
basic gaussian

\end{fulllineitems}

\index{gauss\_dataset() (in module platelib.fitfun)}

\begin{fulllineitems}
\phantomsection\label{\detokenize{platelib:platelib.fitfun.gauss_dataset}}\pysiglinewithargsret{\sphinxcode{\sphinxupquote{platelib.fitfun.}}\sphinxbfcode{\sphinxupquote{gauss\_dataset}}}{\emph{params}, \emph{i}, \emph{x}}{}
calc gaussian from params for data set i
using simple, hardwired naming convention

\end{fulllineitems}

\index{linear() (in module platelib.fitfun)}

\begin{fulllineitems}
\phantomsection\label{\detokenize{platelib:platelib.fitfun.linear}}\pysiglinewithargsret{\sphinxcode{\sphinxupquote{platelib.fitfun.}}\sphinxbfcode{\sphinxupquote{linear}}}{\emph{t}, \emph{a}, \emph{b}}{}
\end{fulllineitems}

\index{objective() (in module platelib.fitfun)}

\begin{fulllineitems}
\phantomsection\label{\detokenize{platelib:platelib.fitfun.objective}}\pysiglinewithargsret{\sphinxcode{\sphinxupquote{platelib.fitfun.}}\sphinxbfcode{\sphinxupquote{objective}}}{\emph{params}, \emph{x}, \emph{data}}{}
calculate total residual for fits to several data sets held
in a 2-D array, and modeled by Gaussian functions

\end{fulllineitems}

\index{quadratic() (in module platelib.fitfun)}

\begin{fulllineitems}
\phantomsection\label{\detokenize{platelib:platelib.fitfun.quadratic}}\pysiglinewithargsret{\sphinxcode{\sphinxupquote{platelib.fitfun.}}\sphinxbfcode{\sphinxupquote{quadratic}}}{\emph{t}, \emph{a}, \emph{b}, \emph{c}}{}
\end{fulllineitems}

\index{sigmoid() (in module platelib.fitfun)}

\begin{fulllineitems}
\phantomsection\label{\detokenize{platelib:platelib.fitfun.sigmoid}}\pysiglinewithargsret{\sphinxcode{\sphinxupquote{platelib.fitfun.}}\sphinxbfcode{\sphinxupquote{sigmoid}}}{\emph{x}, \emph{y0}, \emph{L}, \emph{k}, \emph{x\_half}}{}
\end{fulllineitems}

\index{sigmoidal\_auto() (in module platelib.fitfun)}

\begin{fulllineitems}
\phantomsection\label{\detokenize{platelib:platelib.fitfun.sigmoidal_auto}}\pysiglinewithargsret{\sphinxcode{\sphinxupquote{platelib.fitfun.}}\sphinxbfcode{\sphinxupquote{sigmoidal\_auto}}}{\emph{t}, \emph{a}, \emph{b}, \emph{k}}{}
\end{fulllineitems}



\subsection{platelib.plateread module}
\label{\detokenize{platelib:platelib-plateread-module}}\label{\detokenize{platelib:module-platelib.plateread}}\index{platelib.plateread (module)}\index{Plate\_data (class in platelib.plateread)}

\begin{fulllineitems}
\phantomsection\label{\detokenize{platelib:platelib.plateread.Plate_data}}\pysiglinewithargsret{\sphinxbfcode{\sphinxupquote{class }}\sphinxcode{\sphinxupquote{platelib.plateread.}}\sphinxbfcode{\sphinxupquote{Plate\_data}}}{\emph{data}, \emph{replicates=3}, \emph{rep\_direction='hori'}, \emph{multi\_chrom=1}}{}
Class for containing data from a platereader assay.
\begin{quote}\begin{description}
\item[{Parameters}] \leavevmode\begin{itemize}
\item {} 
\sphinxstyleliteralstrong{\sphinxupquote{data}} \textendash{} A pandas DataFrame with time points as index and wells as columns

\item {} 
\sphinxstyleliteralstrong{\sphinxupquote{replicates}} \textendash{} Positive integer of replicates, assuming equal number of replicates of all samples

\end{itemize}

\end{description}\end{quote}
\index{plot() (platelib.plateread.Plate\_data method)}

\begin{fulllineitems}
\phantomsection\label{\detokenize{platelib:platelib.plateread.Plate_data.plot}}\pysiglinewithargsret{\sphinxbfcode{\sphinxupquote{plot}}}{\emph{titles=None}, \emph{sharey=True}, \emph{plot\_multi=True}}{}
Plots the number the number of sample i.e. replicates/wells in the data 
set.
\begin{quote}\begin{description}
\item[{Parameters}] \leavevmode\begin{itemize}
\item {} 
\sphinxstyleliteralstrong{\sphinxupquote{titles}} \textendash{} List-like object of subtitles

\item {} 
\sphinxstyleliteralstrong{\sphinxupquote{sharey}} \textendash{} Boolean, default is True, where all y-axis limits will be identical. If False y-axis limits per plot are given by matplotlib defaults.

\item {} 
\sphinxstyleliteralstrong{\sphinxupquote{plot\_multi}} \textendash{} Boolean, default is False, is several different measurements are present, only plot the first one. If False plots all of the values.

\end{itemize}

\end{description}\end{quote}

\end{fulllineitems}

\index{to\_a\_csv() (platelib.plateread.Plate\_data method)}

\begin{fulllineitems}
\phantomsection\label{\detokenize{platelib:platelib.plateread.Plate_data.to_a_csv}}\pysiglinewithargsret{\sphinxbfcode{\sphinxupquote{to\_a\_csv}}}{\emph{path}, \emph{one\_per\_multi\_c=False}}{}
Returns the data as a pandas dataframe with times as indexes
\begin{quote}\begin{description}
\item[{Parameters}] \leavevmode\begin{itemize}
\item {} 
\sphinxstyleliteralstrong{\sphinxupquote{path}} \textendash{} String of path to store output

\item {} 
\sphinxstyleliteralstrong{\sphinxupquote{one\_per\_multi\_c}} \textendash{} Boolean, if ‘True’ one measurement per dataframe will be exported otherwise all will be exported in one file.

\end{itemize}

\end{description}\end{quote}

\end{fulllineitems}

\index{to\_a\_dataframe() (platelib.plateread.Plate\_data method)}

\begin{fulllineitems}
\phantomsection\label{\detokenize{platelib:platelib.plateread.Plate_data.to_a_dataframe}}\pysiglinewithargsret{\sphinxbfcode{\sphinxupquote{to\_a\_dataframe}}}{\emph{one\_per\_multi\_c=False}}{}
Returns the data as a list of pandas dataframe(s) with times as indexes
\begin{quote}\begin{description}
\item[{Parameters}] \leavevmode
\sphinxstyleliteralstrong{\sphinxupquote{one\_per\_multi\_c}} \textendash{} Boolean, if ‘True’ one measurement per dataframe will be exported otherwise all will be exported in one dataframe.

\end{description}\end{quote}

\end{fulllineitems}


\end{fulllineitems}

\index{read\_plate() (in module platelib.plateread)}

\begin{fulllineitems}
\phantomsection\label{\detokenize{platelib:platelib.plateread.read_plate}}\pysiglinewithargsret{\sphinxcode{\sphinxupquote{platelib.plateread.}}\sphinxbfcode{\sphinxupquote{read\_plate}}}{\emph{filename}, \emph{replicates=3}, \emph{rep\_direction='hori'}, \emph{time\_unit='hours'}, \emph{named\_samples={[}{]}}, \emph{platereader='bmg'}, \emph{transposed=True}}{}
Reads in data from a CSV file from a BMG or Tecan platereader and returns a platedata object.
\begin{quote}\begin{description}
\item[{Parameters}] \leavevmode\begin{itemize}
\item {} 
\sphinxstyleliteralstrong{\sphinxupquote{filename}} \textendash{} Path to filename as a string

\item {} 
\sphinxstyleliteralstrong{\sphinxupquote{replicates}} \textendash{} The number of replicates per sample, expects a positive integer.

\item {} 
\sphinxstyleliteralstrong{\sphinxupquote{rep\_direction}} \textendash{} The directions replicates is in. Only ‘hori’ and ‘vert’ are accepted directions, ‘hori’ if replicates are going from left to rigth ‘vert’ from replicates going from top to bottom.

\item {} 
\sphinxstyleliteralstrong{\sphinxupquote{time\_unit}} \textendash{} The time unit one would like to have, accepted values are: ‘seconds’,’minutes’,’hours’,’days’

\item {} 
\sphinxstyleliteralstrong{\sphinxupquote{platereader}} \textendash{} The plate reader used to collect the data. Only ‘bmg’ and ‘tecan’ are accepted platereaders

\item {} 
\sphinxstyleliteralstrong{\sphinxupquote{transposed}} \textendash{} Wether the wells are in column (True) or row format (False).

\end{itemize}

\end{description}\end{quote}

\end{fulllineitems}

\index{read\_tecan() (in module platelib.plateread)}

\begin{fulllineitems}
\phantomsection\label{\detokenize{platelib:platelib.plateread.read_tecan}}\pysiglinewithargsret{\sphinxcode{\sphinxupquote{platelib.plateread.}}\sphinxbfcode{\sphinxupquote{read\_tecan}}}{\emph{filename}}{}
Reads in untransposed data from a tecan platereader and returns a pandas DataFrame object.
\begin{quote}\begin{description}
\item[{Parameters}] \leavevmode
\sphinxstyleliteralstrong{\sphinxupquote{filename}} \textendash{} Path to filename as a string

\end{description}\end{quote}

\end{fulllineitems}

\index{read\_transposed\_bmg() (in module platelib.plateread)}

\begin{fulllineitems}
\phantomsection\label{\detokenize{platelib:platelib.plateread.read_transposed_bmg}}\pysiglinewithargsret{\sphinxcode{\sphinxupquote{platelib.plateread.}}\sphinxbfcode{\sphinxupquote{read\_transposed\_bmg}}}{\emph{filename}}{}
Reads in transposed data from a BMG platereader and returns a pandas DataFrame object.
\begin{quote}\begin{description}
\item[{Parameters}] \leavevmode
\sphinxstyleliteralstrong{\sphinxupquote{filename}} \textendash{} Path to filename as a string

\end{description}\end{quote}

\end{fulllineitems}

\index{read\_untransposed\_bmg() (in module platelib.plateread)}

\begin{fulllineitems}
\phantomsection\label{\detokenize{platelib:platelib.plateread.read_untransposed_bmg}}\pysiglinewithargsret{\sphinxcode{\sphinxupquote{platelib.plateread.}}\sphinxbfcode{\sphinxupquote{read\_untransposed\_bmg}}}{\emph{filename}}{}
Reads in untransposed data from a BMG platereader and returns a pandas DataFrame object.
\begin{quote}\begin{description}
\item[{Parameters}] \leavevmode
\sphinxstyleliteralstrong{\sphinxupquote{filename}} \textendash{} Path to filename as a string

\end{description}\end{quote}

\end{fulllineitems}

\index{search\_start() (in module platelib.plateread)}

\begin{fulllineitems}
\phantomsection\label{\detokenize{platelib:platelib.plateread.search_start}}\pysiglinewithargsret{\sphinxcode{\sphinxupquote{platelib.plateread.}}\sphinxbfcode{\sphinxupquote{search\_start}}}{\emph{filename}}{}
Find start of data region and returns the line number by finding the line that starts with “Well”.
\begin{quote}\begin{description}
\item[{Parameters}] \leavevmode
\sphinxstyleliteralstrong{\sphinxupquote{filename}} \textendash{} Path to filename as a string

\end{description}\end{quote}

\end{fulllineitems}

\index{vert\_order() (in module platelib.plateread)}

\begin{fulllineitems}
\phantomsection\label{\detokenize{platelib:platelib.plateread.vert_order}}\pysiglinewithargsret{\sphinxcode{\sphinxupquote{platelib.plateread.}}\sphinxbfcode{\sphinxupquote{vert\_order}}}{\emph{cols}, \emph{reps}}{}
Helper function to reorder data if vertical replicates were made.
\begin{quote}\begin{description}
\item[{Parameters}] \leavevmode\begin{itemize}
\item {} 
\sphinxstyleliteralstrong{\sphinxupquote{cols}} \textendash{} List-like object of columns names (wells) all assumed to have the form ‘A12’.

\item {} 
\sphinxstyleliteralstrong{\sphinxupquote{reps}} \textendash{} List of positive integer number of replicates.

\end{itemize}

\end{description}\end{quote}

\end{fulllineitems}



\subsection{Module contents}
\label{\detokenize{platelib:module-contents}}\label{\detokenize{platelib:module-platelib}}\index{platelib (module)}

\chapter{Contribute}
\label{\detokenize{contribute:contribute}}\label{\detokenize{contribute::doc}}
Contributions are more than welcome, please raise an issue on the github page
highlighting the bug/extension/compatibilities before doing a pull request.


\section{More tools}
\label{\detokenize{contribute:more-tools}}
Apart from the tools listed in {\hyperref[\detokenize{install:installation}]{\sphinxcrossref{\DUrole{std,std-ref}{Installation}}}} the following is needed:
\begin{itemize}
\item {} 
Unix-like system

\item {} 
\sphinxhref{https://git-scm.com/downloads}{git}

\item {} 
\sphinxhref{https://pandoc.org/installing.html}{pandoc}

\end{itemize}


\section{Building}
\label{\detokenize{contribute:building}}\label{\detokenize{contribute:git}}
You have made some wicked cool changes to the source code
or the documentation that you want to share with the world, awesome!

Now there’s just a few steps before they can be incorporated into
the platelib master branch


\subsection{Changing the version number}
\label{\detokenize{contribute:changing-the-version-number}}
The versioning scheme of platelib should be done in reasonable
accordance with the so called \sphinxhref{https://semver.org/}{Semantic versioning} where
X.Y.Z should be read as MAJOR.MINOR.PATCH.

The version number has to be changed in the two files \sphinxcode{\sphinxupquote{setup.py}} and
\sphinxcode{\sphinxupquote{docs/source/conf.py}}


\subsection{Create new source distribution}
\label{\detokenize{contribute:semantic-versioning}}\label{\detokenize{contribute:create-new-source-distribution}}
Go to the docs folder and run:

\fvset{hllines={, ,}}%
\begin{sphinxVerbatim}[commandchars=\\\{\}]
\PYG{o}{.}\PYG{o}{/}\PYG{n}{full\PYGZus{}make}\PYG{o}{.}\PYG{n}{sh}
\end{sphinxVerbatim}

If no errors occurred it can be uploaded to your local branch and
a pull request can be made.


\section{Planned improvements}
\label{\detokenize{contribute:planned-improvements}}
This is as much a wish-list as literally planned improvements:
\begin{itemize}
\item {} 
Plotting
\begin{itemize}
\item {} 
Plotting of data from several plates in some sensible way.

\end{itemize}

\item {} 
Fitting
\begin{itemize}
\item {} 
Local fitting of traces in plate

\item {} 
Global fitting of traces in plate

\end{itemize}

\item {} 
Statistical analysis
\begin{itemize}
\item {} 
Goodness-of-fit

\item {} 
Variance along traces, among replicates, and between conditions

\end{itemize}

\item {} 
Python 3.X compatibility

\item {} 
PyPI availability

\end{itemize}


\renewcommand{\indexname}{Python Module Index}
\begin{sphinxtheindex}
\def\bigletter#1{{\Large\sffamily#1}\nopagebreak\vspace{1mm}}
\bigletter{p}
\item {\sphinxstyleindexentry{platelib}}\sphinxstyleindexpageref{platelib:\detokenize{module-platelib}}
\item {\sphinxstyleindexentry{platelib.fitfun}}\sphinxstyleindexpageref{platelib:\detokenize{module-platelib.fitfun}}
\item {\sphinxstyleindexentry{platelib.plateread}}\sphinxstyleindexpageref{platelib:\detokenize{module-platelib.plateread}}
\end{sphinxtheindex}

\renewcommand{\indexname}{Index}
\printindex
\end{document}